\documentclass[11pt]{article}

\usepackage{times}
\usepackage[english]{babel}

% -----------------------------------------------
% especially use this for you code
% -----------------------------------------------

\usepackage{courier}
\usepackage{listings}
\usepackage{color}
\usepackage{tabularx}
\usepackage{graphicx}

\definecolor{Gray}{gray}{0.95}

\definecolor{mygreen}{rgb}{0,0.6,0}
\definecolor{mygray}{rgb}{0.5,0.5,0.5}
\definecolor{mymauve}{rgb}{0.58,0,0.82}

\lstset{language=C++,
	basicstyle = \normalsize\ttfamily,   % the size and fonts that are used
	tabsize = 2,                    % sets default tabsize
	breaklines = true,              % sets automatic line breaking
	keywordstyle=\color{blue}\ttfamily,
	stringstyle=\color{red}\ttfamily,
	commentstyle=\color{mygreen}\ttfamily,
	numbers=left,
	keepspaces=true,
	showspaces=false,
	showstringspaces=false,
}

\begin{document}

\title{Programming in C/C++ \\
       Exercises set three: polymorphism
}
\date{\today}
\author{Christiaan Steenkist \\
Jaime Betancor Valado \\
Remco Bos \\
}

\maketitle
\section*{Exercise 15, construct ostream class}
We were tasked to construct an ofstream class with our own buffer. The program should work correctly with the syntaxis in the question.

\subsection*{Code listings}
\lstinputlisting[caption = main.ih]{src/a15/main.ih}
\lstinputlisting[caption = main.cc]{src/a15/main.cc}
\lstinputlisting[caption = bistream.h]{src/a15/bistream.h}
\lstinputlisting[caption = bistreambuffer.h]{src/a15/bistreambuffer.h}
\lstinputlisting[caption = bufferconstructor.cc]{src/a15/bufferconstructor.cc}
\lstinputlisting[caption = streamconstructor.cc]{src/a15/streamconstructor.cc}
\lstinputlisting[caption = streamdestructor.cc]{src/a15/streamdestructor.cc}
\lstinputlisting[caption = xsputn.cc]{src/a15/xsputn.cc}

\section*{Exercise 16, design streambuf}
We were tasked to design a streambuf class that is called IFdStreamBuff that allows extractions from a FD.

\subsection*{Code listings}
\lstinputlisting[caption = ifdstreambuf.ih]{src/a18/ifdstreambuf.ih}
\lstinputlisting[caption = mode.h]{src/a18/fdbuffermode.h}
\lstinputlisting[caption = ifdstreambuf.h]{src/a18/ifdstreambuf.h}
\lstinputlisting[caption = close.cc]{src/a18/iclose.cc}
\lstinputlisting[caption = cnstr1.cc]{src/a18/iconstr1.cc}
\lstinputlisting[caption = cnstr2.cc]{src/a18/iconstr2.cc}
\lstinputlisting[caption = destructor.cc]{src/a18/idestructor.cc}
\lstinputlisting[caption = open.cc]{src/a18/iopen.cc}
\lstinputlisting[caption = uflow.cc]{src/a18/iuflow.cc}
\lstinputlisting[caption = underflow.cc]{src/a18/iunderflow.cc}
\lstinputlisting[caption = xsgetn.cc]{src/a18/xsgetn.cc}

\section*{Exercise 17, design streambuf 2}
We were tasked to design the OFdStreamBuff that allows insertions to a FD.

\subsection*{Code listings}
\lstinputlisting[caption = ofdstreambuf.ih]{src/a18/ofdstreambuf.ih}
\lstinputlisting[caption = ofdstreambuf.h]{src/a18/ofdstreambuf.h}
\lstinputlisting[caption = close.cc]{src/a18/oclose.cc}
\lstinputlisting[caption = cnstr1.cc]{src/a18/oconstr1.cc}
\lstinputlisting[caption = cnstr2.cc]{src/a18/oconstr2.cc}
\lstinputlisting[caption = destructor.cc]{src/a18/odestructor.cc}
\lstinputlisting[caption = open.cc]{src/a18/oopen.cc}
\lstinputlisting[caption = overflow.cc]{src/a18/ooverflow.cc}
\lstinputlisting[caption = psync.cc]{src/a18/psync.cc}
\lstinputlisting[caption = sync.cc]{src/a18/sync.cc}
\lstinputlisting[caption = xsputn.cc]{src/a18/xsputn.cc}

\section*{Exercise 18: FD streams}
Here is the code for the streams of the corresponding FD buffers from exercises 16 and 17.
The code in main echoes back whatever is typed into console.

\subsection*{Code listings}
\lstinputlisting[caption = main.h]{src/a18/main.h}
\lstinputlisting[caption = main.cc]{src/a18/main.cc}

\subsubsection*{\texttt{iFdStreambuf}}
\lstinputlisting[caption = ifdstream.h]{src/a18/ifdstream.h}
\lstinputlisting[caption = istreamconstr.cc]{src/a18/istreamconstr.cc}
\lstinputlisting[caption = ostreamconstr.cc]{src/a18/istreamdestr.cc}

\subsubsection*{\texttt{oFdStreambuf}}
\lstinputlisting[caption = ofdstream.h]{src/a18/ofdstream.h}
\lstinputlisting[caption = istreamdestr.cc]{src/a18/ostreamconstr.cc}
\lstinputlisting[caption = ostreamdestr.cc]{src/a18/ostreamdestr.cc}

\section*{Exercise 19: Forks}
We were tasked with making an abstract \texttt{Fork}.
Its derived classes are able to fork themselves by calling the member \texttt{fork()}.
A tester class was made to check if the forking works.

\subsection*{Sample output}
\begin{lstlisting}
Parent process 26432 here!
BEEP
Child process 26433 here!
BOOP
\end{lstlisting}

\subsection*{Code listings}
\lstinputlisting[caption = main.cc]{src/a19/main.cc}
\lstinputlisting[caption = fork.ih]{src/a19/fork/fork.ih}
\lstinputlisting[caption = fork.h]{src/a19/fork/fork.h}
\lstinputlisting[caption = childprocesstester.cc]{src/a19/fork/childprocesstester.cc}
\lstinputlisting[caption = forkdestructor.cc]{src/a19/fork/forkdestructor.cc}
\lstinputlisting[caption = parentprocesstester.cc]{src/a19/fork/parentprocesstester.cc}
\lstinputlisting[caption = pid.cc]{src/a19/fork/pid.cc}
\lstinputlisting[caption = testdestructor.cc]{src/a19/fork/testdestructor.cc}
\lstinputlisting[caption = waitforchild.cc]{src/a19/fork/waitforchild.cc}
\end{document}
